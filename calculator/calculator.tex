\documentclass[nocopyrightspace]{sigplanconf}

\usepackage{bcprules}
\usepackage{amsmath}
\usepackage{amsfonts}
\usepackage{amssymb}
\usepackage{amsthm}
\usepackage[mathscr]{eucal}
\usepackage{stmaryrd}
\usepackage{xspace}
\usepackage{mathrsfs}
\usepackage{txfonts}
\usepackage{array}
\usepackage{xargs}
\usepackage[figuresleft]{rotating}
\usepackage{color}
\usepackage{mdframed}

%% misc
\newcommand{\alt}{\ \mid\ }
\newcommand{\lalt}{\ \ \ \alt}
\newcommand{\anywhere}[1]{\ensuremath{\mbox{#1}}}
\newcommand{\kw}[1]{\anywhere{\sffamily{\bfseries\small {#1}}}}
\newcommand{\red}[1]{{\color{red}{#1}}}
\newcommand{\blue}[1]{{\color{blue}{#1}}}
\newcommand{\green}[1]{{\color{green}{#1}}}
\newcommand{\note}[1]{\red{\textbf{#1}}}
\newcommand{\ignore}[1]{}
\newcommand{\ulist}[1]{\begin{itemize} #1 \end{itemize}}
\newcommand{\vsp}{\vspace{.1in}}
\newcommand{\nvsp}{\vspace{-.1in}}
\newcommandx{\framed}[2][1=\nvsp\nvsp]{\begin{mdframed}#1 #2 \nvsp\nvsp\end{mdframed}}
%\newcommandx{\framed}[2][1=\nvsp\nvsp]{#2}
\newcommand{\say}{\vsp\vsp\noindent}

%% fonts
\newcommand{\mtt}[1]{\ensuremath{\mathit{#1}}}
\newcommand{\mrm}[1]{\ensuremath{\mathrm{#1}}}
\newcommand{\ttt}[1]{\ensuremath{\mbox{\small \texttt{#1}}}}

%% syntactic domains
\newcommand{\Variable}{\ensuremath{\mtt{Variable}}\xspace}
\newcommand{\Exp}{\ensuremath{\mtt{Expression}}\xspace}

%% syntactic objects
\newcommand{\set}[1]{\ensuremath{\overrightarrow{#1}}}
\newcommandx{\seq}[1][1=e]{\ensuremath{\vec{#1}}}
\newcommandx{\fun}[2][1=x, 2=e]{\ensuremath{\lambda #1.#2}}

%% semantic domains
\newcommand{\Env}{\ensuremath{\mtt{Env}}\xspace}
\newcommand{\Store}{\ensuremath{\mtt{Store}}\xspace}
\newcommand{\BaseValue}{\ensuremath{\mtt{BaseValue}}\xspace}
\newcommand{\JumpValue}{\ensuremath{\mtt{JumpValue}}\xspace}
\newcommand{\FEIValue}{\ensuremath{\mtt{FEIValue}}\xspace}
\newcommand{\FEOValue}{\ensuremath{\mtt{FEOValue}}\xspace}
\newcommand{\Value}{\ensuremath{\mtt{Value}}\xspace}
\newcommand{\Addr}{\ensuremath{\mtt{Address}}\xspace}
\newcommand{\Conf}{\ensuremath{\mtt{Config}}\xspace}
\newcommand{\Closure}{\ensuremath{\mtt{Closure}}\xspace}
\newcommand{\List}{\ensuremath{\mtt{List}}\xspace}

%% semantic objects
\newcommand{\conf}{\ensuremath{\mathcal{C}}}
\newcommand{\env}{\ensuremath{\rho}}
\newcommand{\store}{\ensuremath{\sigma}}
\newcommand{\eval}{\ensuremath{\Downarrow}}
\newcommand{\dom}{\ensuremath{\mtt{dom}}\xspace}
\newcommand{\init}{\ensuremath{\ttt{inject}}\xspace}
\newcommand{\clo}{\ensuremath{\mtt{clo}}\xspace}
\newcommand{\clov}[2]{\ensuremath{(#1 \cdot #2)}}

% misc
\newcommandx{\listtwo}[2][1=e_1, 2=e_2]{#1\cdot#2}
\newcommandx{\listthree}[3][1=e_1, 2=e_2, 3=e_3]{#1\cdot#2\cdot#3}

\newcommand{\pair}[2]{\ensuremath{(#1,\, #2)}}
\newcommand{\feinput}{\set{\pair{g}{\store}}}
\newcommand{\ruleBlock}{\vsp\vsp\vsp}
\newcommand{\subscr}[1]{\textsubscript{#1}}
\newcommandx{\nth}[1][1=n]{\ensuremath{#1}th\xspace}
%\newcommand{\assign}[2]{#1 \triangleq #2}
\newcommand{\assign}[2]{#1 = #2}
\newcommand{\assignRev}[2]{\assign{#2}{#1}}
\newcommand{\typeError}{``\mtt{TypeError}"}
\newcommand{\notImpl}{``\mtt{NotImplemented}"}
\newcommand{\todo}{\red{TODO:}\xspace}
\newcommand{\Int}{\ensuremath{\mathbb{Z}}\xspace}
\newcommand{\BinOp}{\ensuremath{\mtt{BinaryOperation}}\xspace}
\newcommand{\bop}{\ensuremath{\oplus}}



\begin{document}
\title{Calculator}
\authorinfo{}{}{}
\date{}
\maketitle

\section{Background}
This document describes the \emph{syntax} and \emph{semantics} of a simple language.
A language's \emph{syntax} refers to what we can write in the language, and the \emph{semantics} refer to what the language does in response to what is written.
These two ideas are distinct, despite the fact that they are often conflated with each other.

In this document, the language of interest is that of a simple evaluator of arithmetic expressions.
That is, given the expression $2 + 3$, we define a series of rules to get back the answer $5$.
The point here is to take something you are already familiar with and show what it looks like when formalized.

A caveat for someone new to programming is that, pretty much no matter what we do, somewhere there is some idea of evaluation.
This is annoying, considering that programming is really the art of manipulating evaluation.
I have yet to come up with a better way; at some point, we need evaluation somewhere.
I have tried to make the steps as simple as possible to avoid any jumps in logic.

\section{Concrete Syntax}
For most any language, there is a notion of \emph{concrete syntax}.
The concrete syntax is what the user actually writes down.
For our calculator, some example expressions follow:
\begin{align*}
  &3 \times 7 + 23\\
  &42 \div  2 \times 4 - 2 \times 9\\
\end{align*}

\section{Abstract Syntax}
The \emph{abstract syntax} of a language is similar to the concrete syntax, except certain things are abstracted away.
One of the most important things that is commonly abstracted away is operator precidence.
For example, consider the arithmetic expression $2 + 3 \times 7$.
Based on the PEMDAS rules, we know that we should look at this expression with an extra set of parentheses around the $\times$, like so: $2 + (3 \times 7)$.
Handling these sorts of things can get tricky, and it tends to be a separate concern (who decided that $\times$ has higher precedence than $+$, anyway?).
In fact, these sort of issues can be encapsulated in the \emph{parser}, a tool that reads in some input string and converts it to abstract syntax.

For our purposes, the job of the parser is not particularly interesting.
As such, we tend to work only with abstract syntax.

Abstract syntax is usually represented as rules in \emph{Backus-Naur Form}, or BNF for short.
BNF defines what constitutes a \emph{syntactically well-formed} program.
That is, given some input, we can use BNF rules to determine if something has the syntax of a program.
BNF can be seen as a way to succintly describe all possible programs in a given language, which is impressive considering that nearly all languages have an infinite number of such programs.
A full description of BNF is beyond the scope of this document, but what is contained within should be enough to get you rolling.

The BNF rules corresponding to our expression evaluator follow.
\framed[]{
  \begin{gather*}
    n \in \Int
  \end{gather*}
}
The rule above states that $n$ is some member of $\Int$, i.e., the infinite set of all integers.
In plain English, this means that everytime we see a use of $n$, this is a stand-in for some integer.

Now for the operations we will use:
\framed[]{
  \begin{align*}
    \bop \in \BinOp &::= + | - | \times | \div
  \end{align*}
}

What the above says is that anytime we see $\bop$, this can be one of $+$, $-$, $\times$, or $\div$.
The name $\BinOp$, and the use of the identifier $\bop$, are not really important for understanding what this does.
All were defining at this point is what this language looks like - the syntax, not the semantics.

Now for the actual arithmetic expressions:
\framed[]{
  \begin{align*}
    e \in \Exp &::= n \alt e_1 \bop e_2
  \end{align*}
}

The above rule states that a syntactically well-formed $\Exp$ (whatever that is) is either an integer ($n$) or some other expression ($e_1$) followed by an operation ($\bop$) and a second expression ($e_2$).
To show this in action, consider the expression $(3 \times 7) + 23$.
We can view this as $e_1 + e_2$, where $e_1 = 3 \times 7$ and $e_2 = 23$.
Going further, $3 \times 7$, AKA $e_1$, can be viewed as $e_3 \times e_4$, where $e_3 = 3$ and $e_4 = 7$.
From here, $e_2$, $e_3$, and $e_4$ are all integers, and so we can confirm that this expression is syntactically well-formed.

Going back to the difference between concrete and abstract syntax, with the above example, the parentheses were put in place already by the parser.
As shown, because of the parentheses, there was no ambiguity in figuring out how to look at the same expression through the eyes of abstract syntax.

\section{Semantics}
At this point, we've defined the abstract syntax of our language.
Now we need to attach a semantics to this syntax.
There are a variety of accepted ways to do this, all with specific pros and cons.
One approach is to talk about it informally in English, but English is not a particularly precise or concise language for this.
Instead, we will take a formal approach rooted in mathematics, and define what is known as a \emph{big-step semantics}.
In a big-step semantics, we define a \emph{recursive evaluator} over the abstract syntax.
It is an \emph{evaluator} in that it evaluates some expression down to a value.
It is \emph{recursive} in that it is defined largely in terms of itself.
For example, consider the evaluation of some arithmetic expression $e_1 + e_2$.
Here, we could (and will) define this by evaluating $e_1$ and then evaluating $e_2$, and finally adding the results together.
This is actually recursive, because both $e_1$ and $e_2$ could themselves have the form $e_3 + e_4$, and so on.
By the nature of the BNF rules, we are guaranteed that we will eventually hit a number, and so we won't go about this same process forever.
That is, while the BNF rules describe an infinite set of programs, each program is of finite size.

Without further ado, let's get to the semantics.
First, we define exactly what it is our evaluator does:

\framed[]{
  \begin{gather*}
    \eval \;\in \Exp \to \Int
  \end{gather*}
}

This is known as a \emph{signature}: a definition of the \emph{types} an operation works on.
The above signature states that $\eval$ takes an arithmetic expression (which is a type) and gives us back an integer (which is another type).
Intuitively, this works just like a standard desk calculator; the user types in an arithmetic expression, and the calculator gives back an answer.
For simplicity, we consider a calculator that can handle only integers, but the whole formalization is pretty much the same if we were to apply this to, say, real numbers.

Now that we have the signature of $\eval$, we can start writing some rules for it.
The first one is below:

\framed[]{
  \infax[num]
  {
    n \eval n
  }
}

What the above rule states is that if we give some integer $n$ to $\eval$ (the item to the left of the $\eval$), then we get back the same integer $n$.

Now for another rule.
Before trying to read this rule, we define the operator $\bar{\bop}$ to correspond to some predefined arithmetic operator, one that takes two \textbf{integers} (not expressions!) and returns an integer.
For example, $\bar{+}$ corresponds to some given arithmetic operator for $+$.
Without further ado, here is the rule:

\framed[]{
  \infrule[op-norm]
  {
    \bop \in \{ +, -, \times \} \andalso e_1 \eval n_1 \andalso e_2 \eval n_2 \andalso n = n_1 \;\bar{\bop}\; n_2
  }
  {
    e_1 \bop e_2 \eval n
  }
}

Suddenly things got a lot more complex.
To read this rule, first look at what is below the big line, specifically that to the left of the $\eval$.
Looking only at that portion, in plain English, this rule is stating that it applies whenever the program has the form $e_1 \bop e_2$.
From there, the rule manipulates $e_1$, $\bop$, and $e_2$ in specific ways.

For any rule with a line like this, anything above the line must also hold for the rule as a whole to hold.
Each chunk above the bar (e.g., the $\bop \in \{ +, -, \times \}$ and $e_1 \eval n_1$ portions) is referred to as a \emph{premise}.
A premise is a logical statement that can be either true or false, and can manipulate variables like $e_1$ and $\bop$ in the process.

Let's go over each one of these premises now.
The first premise says that for this rule to apply, the arithmetic operator being applied (i.e., $\bop$) must be one of $+$, $-$, or $\times$.
This rule does not work for $\div$; that will be handled in a later rule.

The second premise says that the expression $e_1$ evaluates down to the integer $n_1$.
Logically, this holds as long as $e_1$ can be evaluated.

The third premise is much like the second, except that it is for the expression $e_2$ and the integer $n_2$.

Now for the fourth premise, which has a lot of buried complexity.
This is stating, in plain English, that some new variable $n$ is equal to the result of $n_1 \;\bar{\bop}\; n_2$.
Recall that $\bar{\bop}$ refers to the underlying arithmetic operation for $\bop$.
Basically, this just lets us perform the actual operation, assuming a form of each arithmetic operator is available.
For arithmetic specifically, it is commonly assumed that such operations exist, as they usually do when we go to implement these rules on a computer.
We could go through the trouble of implementing these operators right in the semantics themselves, but this is usually considered bad practice because:
\begin{enumerate}
  \item People generally understand what these operations must do anyway.
    The point of devising a formal semantics specification is to explain anything that might be tricky, not to go through tedium for the sake of tedium.
    Adding them would simply clutter up things and make them needlessly complex.
  \item They constrain what it means to be a ``correct'' implementation in a very bad way.
    For example, say we have a particularly inefficient way of handling multiplication specified in the semantics.
    For an implementation to be completely correct and consistent, it \textbf{must} implement this inefficient way, even if there is some other way available which is better in every way imaginable.
\end{enumerate}

So what about the $\div$ operator?
For this, we have one last rule, just for $\div$:

\framed[]{
  \infrule[op-div]
  {
    e_1 \eval n_1 \andalso e_2 \eval n_2 \andalso n_2 \neq 0 \andalso n = n_1 \;\bar{\div}\; n_2
  }
  {
    e_1 \div e_2 \eval n
  }
}

This time around, there is no premise to constrain $\bop$; $\bop$ was already constrainted to be $\div$ based on what is to the left of $\eval$ below the bar.
The first and second premises are identical to the second and third premises of the previous rule; we just need to evaluate $e_1$ and $e_2$ down to the integers $n_1$ and $n_2$, respectively.
The third premise is where things start to get interesting.
In the third premise, we constrain this rule to only apply when $n_2 \neq 0$.
In plain English, this says that we don't allow division by zero.
The fourth premise performs the actual division, and is guaranteed to never see a division by zero thanks to the third premise.

The above rule is that the last one we will define.
This begs the question then - what happens if we attempt to divide by zero?
In this case, no rule applies.
Whenever we hit a case where no rule applies, the technical term is \emph{getting stuck} (seriously), and it usually indicates some sort of error condition.
When we go to implement this on a computer, we may want to present the user with an error of some sort.

Now for a thought question.
What happens if we were to remove the first premise from OP-NORM (the rule before division)?
That is, we allow this rule to operate on more than just $+$, $-$, and $\times$.
In this case, the semantics would become \emph{nondeterministic}, in constract with the current \emph{deterministic} semantics.
In a deterministic semantics, at any given point in time, at most one rule applies, whereas a nondeterministic allows for multiple rules at a time to apply.
Mathematically, there is nothing wrong with this - one can think of splitting off computation in every direction where a rule applies, and then performing all subsequent computations in parallel.
When it comes to actually implementing things though, a nondeterministic semantics is usually undesirable - this means that the program can do mean different things at the same time!
There are some languages for which this makes sense, but oftentimes we need to be careful to make our rules mutually exclusive.

\end{document}
