\documentclass[nocopyrightspace]{sigplanconf}

\usepackage{bcprules}
\usepackage{amsmath}
\usepackage{amsfonts}
\usepackage{amssymb}
\usepackage{amsthm}
\usepackage[mathscr]{eucal}
\usepackage{stmaryrd}
\usepackage{xspace}
\usepackage{mathrsfs}
\usepackage{txfonts}
\usepackage{array}
\usepackage{xargs}
\usepackage[figuresleft]{rotating}
\usepackage{color}
\usepackage{mdframed}

%% misc
\newcommand{\alt}{\ \mid\ }
\newcommand{\lalt}{\ \ \ \alt}
\newcommand{\anywhere}[1]{\ensuremath{\mbox{#1}}}
\newcommand{\kw}[1]{\anywhere{\sffamily{\bfseries\small {#1}}}}
\newcommand{\red}[1]{{\color{red}{#1}}}
\newcommand{\blue}[1]{{\color{blue}{#1}}}
\newcommand{\green}[1]{{\color{green}{#1}}}
\newcommand{\note}[1]{\red{\textbf{#1}}}
\newcommand{\ignore}[1]{}
\newcommand{\ulist}[1]{\begin{itemize} #1 \end{itemize}}
\newcommand{\vsp}{\vspace{.1in}}
\newcommand{\nvsp}{\vspace{-.1in}}
\newcommandx{\framed}[2][1=\nvsp\nvsp]{\begin{mdframed}#1 #2 \nvsp\nvsp\end{mdframed}}
%\newcommandx{\framed}[2][1=\nvsp\nvsp]{#2}
\newcommand{\say}{\vsp\vsp\noindent}

%% fonts
\newcommand{\mtt}[1]{\ensuremath{\mathit{#1}}}
\newcommand{\mrm}[1]{\ensuremath{\mathrm{#1}}}
\newcommand{\ttt}[1]{\ensuremath{\mbox{\small \texttt{#1}}}}

%% syntactic domains
\newcommand{\Variable}{\ensuremath{\mtt{Variable}}\xspace}
\newcommand{\Exp}{\ensuremath{\mtt{Expression}}\xspace}

%% syntactic objects
\newcommand{\set}[1]{\ensuremath{\overrightarrow{#1}}}
\newcommandx{\seq}[1][1=e]{\ensuremath{\vec{#1}}}
\newcommandx{\fun}[2][1=x, 2=e]{\ensuremath{\lambda #1.#2}}

%% semantic domains
\newcommand{\Env}{\ensuremath{\mtt{Env}}\xspace}
\newcommand{\Store}{\ensuremath{\mtt{Store}}\xspace}
\newcommand{\BaseValue}{\ensuremath{\mtt{BaseValue}}\xspace}
\newcommand{\JumpValue}{\ensuremath{\mtt{JumpValue}}\xspace}
\newcommand{\FEIValue}{\ensuremath{\mtt{FEIValue}}\xspace}
\newcommand{\FEOValue}{\ensuremath{\mtt{FEOValue}}\xspace}
\newcommand{\Value}{\ensuremath{\mtt{Value}}\xspace}
\newcommand{\Addr}{\ensuremath{\mtt{Address}}\xspace}
\newcommand{\Conf}{\ensuremath{\mtt{Config}}\xspace}
\newcommand{\Closure}{\ensuremath{\mtt{Closure}}\xspace}
\newcommand{\List}{\ensuremath{\mtt{List}}\xspace}

%% semantic objects
\newcommand{\conf}{\ensuremath{\mathcal{C}}}
\newcommand{\env}{\ensuremath{\rho}}
\newcommand{\store}{\ensuremath{\sigma}}
\newcommand{\eval}{\ensuremath{\Downarrow}}
\newcommand{\dom}{\ensuremath{\mtt{dom}}\xspace}
\newcommand{\init}{\ensuremath{\ttt{inject}}\xspace}
\newcommand{\clo}{\ensuremath{\mtt{clo}}\xspace}
\newcommand{\clov}[2]{\ensuremath{(#1 \cdot #2)}}

% misc
\newcommandx{\listtwo}[2][1=e_1, 2=e_2]{#1\cdot#2}
\newcommandx{\listthree}[3][1=e_1, 2=e_2, 3=e_3]{#1\cdot#2\cdot#3}

\newcommand{\pair}[2]{\ensuremath{(#1,\, #2)}}
\newcommand{\feinput}{\set{\pair{g}{\store}}}
\newcommand{\ruleBlock}{\vsp\vsp\vsp}
\newcommand{\subscr}[1]{\textsubscript{#1}}
\newcommandx{\nth}[1][1=n]{\ensuremath{#1}th\xspace}
%\newcommand{\assign}[2]{#1 \triangleq #2}
\newcommand{\assign}[2]{#1 = #2}
\newcommand{\assignRev}[2]{\assign{#2}{#1}}
\newcommand{\typeError}{``\mtt{TypeError}"}
\newcommand{\notImpl}{``\mtt{NotImplemented}"}
\newcommand{\todo}{\red{TODO:}\xspace}
\newcommand{\Int}{\ensuremath{\mathbb{Z}}\xspace}
\newcommand{\BinOp}{\ensuremath{\mtt{BinaryOperation}}\xspace}
\newcommand{\bop}{\ensuremath{\oplus}}



\begin{document}
\title{Calculator}
\authorinfo{}{}{}
\date{}
\maketitle

\section{Background}
This document describes the \emph{syntax} and \emph{semantics} of a simple language.
A language's \emph{syntax} refers to what we can write in the language, and the \emph{semantics} refer to what the language does in response to what is written.
These two ideas are distinct, despite the fact that they are often conflated with each other.

In this document, the languagr of interest is that of a simple evaluator of arithmetic expressions.
That is, given the expression $2 + 3$, we define a series of rules to get back the answer $5$.
The point here is to take something you are already familiar with and show what it looks like when formalized.

\section{Concrete Syntax}
For most any language, there is a notion of \emph{concrete syntax}.
The concrete syntax is what the user actually writes down.
For our calculator, some example expressions follow:
\begin{align*}
  &3 \times 7 + 23\\
  &42 \div  2 \times 4 - 2 \times 9\\
\end{align*}

\section{Abstract Syntax}
The \emph{abstract syntax} of a language is similar to the concrete syntax, except certain things are abstracted away.
One of the most important things that is commonly abstracted away is operator precidence.
For example, consider the arithmetic expression $2 + 3 \times 7$.
Based on the PEMDAS rules, we know that we should look at this expression with an extra set of parentheses around the $\times$, like so: $2 + (3 \times 7)$.
Handling these sorts of things can get tricky, and it tends to be a separate concern (who decided that $\times$ has higher precedence than $+$, anyway?).
In fact, these sort of issues can be encapsulated in the \emph{parser}, a tool that reads in some input string and converts it to abstract syntax.

For our purposes, the job of the parser is not particularly interesting.
As such, we tend to work only with abstract syntax.

Abstract syntax is usually represented as rules in \emph{Backus-Naur Form}, or BNF for short.
BNF defines what constitutes a \emph{syntactically well-formed} program.
That is, given some input, we can use BNF rules to determine if something has the syntax of a program.
BNF can be seen as a way to succintly describe all possible programs in a given language, which is impressive considering that nearly all languages have an infinite number of such programs.
A full description of BNF is beyond the scope of this document, but what is contained within should be enough to get you rolling.

The BNF rules corresponding to our expression evaluator follow.
\framed[]{
  \begin{gather*}
    n \in \Int
  \end{gather*}
}
The rule above states that $n$ is some member of $\Int$, i.e., the infinite set of all integers.
In plain English, this means that everytime we see a use of $n$, this is a stand-in for some integer.

Now for the operations we will use:
\framed[]{
  \begin{align*}
    \bop \in \BinOp &::= + | - | \times | \div
  \end{align*}
}

What the above says is that anytime we see $\bop$, this can be one of $+$, $-$, $\times$, or $\div$.
The name $\BinOp$, and the use of the identifier $\bop$, are not really important for understanding what this does.
All were defining at this point is what this language looks like - the syntax, not the semantics.

Now for the actual arithmetic expressions:
\framed[]{
  \begin{align*}
    e \in \Exp &::= n \alt e_1 \bop e_2
  \end{align*}
}

The above rule states that a syntactically well-formed $\Exp$ (whatever that is) is either an integer ($n$) or some other expression ($e_1$) followed by an operation ($\bop$) and a second expression ($e_2$).
To show this in action, consider the expression $(3 \times 7) + 23$.
We can view this as $e_1 + e_2$, where $e_1 = 3 \times 7$ and $e_2 = 23$.
Going further, $3 \times 7$, AKA $e_1$, can be viewed as $e_3 \times e_4$, where $e_3 = 3$ and $e_4 = 7$.
From here, $e_2$, $e_3$, and $e_4$ are all integers, and so we can confirm that this expression is syntactically well-formed.

Going back to the difference between concrete and abstract syntax, with the above example, the parentheses were put in place already by the parser.
As shown, because of the parentheses, there was no ambiguity in figuring out how to look at the same expression through the eyes of abstract syntax.

\section{Semantics}
At this point, we've defined the abstract syntax of our language.
Now we need to attach a semantics to this syntax.

\end{document}
