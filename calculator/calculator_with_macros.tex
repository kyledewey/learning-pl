\documentclass[nocopyrightspace]{sigplanconf}

\usepackage{bcprules}
\usepackage{amsmath}
\usepackage{amsfonts}
\usepackage{amssymb}
\usepackage{amsthm}
\usepackage[mathscr]{eucal}
\usepackage{stmaryrd}
\usepackage{xspace}
\usepackage{mathrsfs}
\usepackage{txfonts}
\usepackage{array}
\usepackage{xargs}
\usepackage[figuresleft]{rotating}
\usepackage{color}
\usepackage{mdframed}

%% misc
\newcommand{\alt}{\ \mid\ }
\newcommand{\lalt}{\ \ \ \alt}
\newcommand{\anywhere}[1]{\ensuremath{\mbox{#1}}}
\newcommand{\kw}[1]{\anywhere{\sffamily{\bfseries\small {#1}}}}
\newcommand{\red}[1]{{\color{red}{#1}}}
\newcommand{\blue}[1]{{\color{blue}{#1}}}
\newcommand{\green}[1]{{\color{green}{#1}}}
\newcommand{\note}[1]{\red{\textbf{#1}}}
\newcommand{\ignore}[1]{}
\newcommand{\ulist}[1]{\begin{itemize} #1 \end{itemize}}
\newcommand{\vsp}{\vspace{.1in}}
\newcommand{\nvsp}{\vspace{-.1in}}
\newcommandx{\framed}[2][1=\nvsp\nvsp]{\begin{mdframed}#1 #2 \nvsp\nvsp\end{mdframed}}
%\newcommandx{\framed}[2][1=\nvsp\nvsp]{#2}
\newcommand{\say}{\vsp\vsp\noindent}

%% fonts
\newcommand{\mtt}[1]{\ensuremath{\mathit{#1}}}
\newcommand{\mrm}[1]{\ensuremath{\mathrm{#1}}}
\newcommand{\ttt}[1]{\ensuremath{\mbox{\small \texttt{#1}}}}

%% syntactic domains
\newcommand{\Variable}{\ensuremath{\mtt{Variable}}\xspace}
\newcommand{\Exp}{\ensuremath{\mtt{Expression}}\xspace}

%% syntactic objects
\newcommand{\set}[1]{\ensuremath{\overrightarrow{#1}}}
\newcommandx{\seq}[1][1=e]{\ensuremath{\vec{#1}}}
\newcommandx{\fun}[2][1=x, 2=e]{\ensuremath{\lambda #1.#2}}

%% semantic domains
\newcommand{\Env}{\ensuremath{\mtt{Env}}\xspace}
\newcommand{\Store}{\ensuremath{\mtt{Store}}\xspace}
\newcommand{\BaseValue}{\ensuremath{\mtt{BaseValue}}\xspace}
\newcommand{\JumpValue}{\ensuremath{\mtt{JumpValue}}\xspace}
\newcommand{\FEIValue}{\ensuremath{\mtt{FEIValue}}\xspace}
\newcommand{\FEOValue}{\ensuremath{\mtt{FEOValue}}\xspace}
\newcommand{\Value}{\ensuremath{\mtt{Value}}\xspace}
\newcommand{\Addr}{\ensuremath{\mtt{Address}}\xspace}
\newcommand{\Conf}{\ensuremath{\mtt{Config}}\xspace}
\newcommand{\Closure}{\ensuremath{\mtt{Closure}}\xspace}
\newcommand{\List}{\ensuremath{\mtt{List}}\xspace}

%% semantic objects
\newcommand{\conf}{\ensuremath{\mathcal{C}}}
\newcommand{\env}{\ensuremath{\rho}}
\newcommand{\store}{\ensuremath{\sigma}}
\newcommand{\eval}{\ensuremath{\Downarrow}}
\newcommand{\dom}{\ensuremath{\mtt{dom}}\xspace}
\newcommand{\init}{\ensuremath{\ttt{inject}}\xspace}
\newcommand{\clo}{\ensuremath{\mtt{clo}}\xspace}
\newcommand{\clov}[2]{\ensuremath{(#1 \cdot #2)}}

% misc
\newcommandx{\listtwo}[2][1=e_1, 2=e_2]{#1\cdot#2}
\newcommandx{\listthree}[3][1=e_1, 2=e_2, 3=e_3]{#1\cdot#2\cdot#3}

\newcommand{\pair}[2]{\ensuremath{(#1,\, #2)}}
\newcommand{\feinput}{\set{\pair{g}{\store}}}
\newcommand{\ruleBlock}{\vsp\vsp\vsp}
\newcommand{\subscr}[1]{\textsubscript{#1}}
\newcommandx{\nth}[1][1=n]{\ensuremath{#1}th\xspace}
%\newcommand{\assign}[2]{#1 \triangleq #2}
\newcommand{\assign}[2]{#1 = #2}
\newcommand{\assignRev}[2]{\assign{#2}{#1}}
\newcommand{\typeError}{``\mtt{TypeError}"}
\newcommand{\notImpl}{``\mtt{NotImplemented}"}
\newcommand{\todo}{\red{TODO:}\xspace}
\newcommand{\Int}{\ensuremath{\mathbb{Z}}\xspace}
\newcommand{\BinOp}{\ensuremath{\mtt{BinaryOperation}}\xspace}
\newcommand{\bop}{\ensuremath{\oplus}}


\usepackage{hyperref}

\begin{document}
\title{Calculator With Macros}
\authorinfo{}{}{}
\date{}
\maketitle

\section{Introduction and Motivation}
\label{sec:intro}
The previous lesson introduced a basic calculator language.
This language is sufficient for expressing all sorts of arithmetic expressions, though it is sorely lacking in a number of features.
One such lacking feature is that of \emph{abstraction}: the language does not provide the programmer any means of separating out parts of the program and using them in mostly independent ways.

To demonstrate, let's say the programmer wants to take two Fahrenheit temperatures, convert them to Celcius, and then subtract the first Celcius temperature from the second.
If the two Fahrenheit temperatures are represented with the variables $f_1$ and $f_2$, then a simple formula for this is the following:
\begin{align*}
{\tt toCelcius}(f_1) - {\tt toCelcius}(f_2)
\end{align*}
\ldots where {\tt toCelcius} is defined as the following formula:
\begin{flalign*}
&{\tt toCelcius} \in \mathbb{R} \to \mathbb{R}&
\\
&{\tt toCelcius}(f) = (f - 32) \times (5 \div 9)&
\end{flalign*}

As shown, the programmer has \emph{abstracted} the formula to convert from Celsius to Fahrenheit into its own \emph{definition}, allowing it to be reused.
Without this, the programmer would be forced to write out the {\tt toCelcius} formula explicitly each time it is used, as with:
\begin{align*}
((f_1 - 32) \times (5 \div 9)) - ((f_2 - 32) \times (5 \div 9))
\end{align*}

There are a number of disadvantages to having to write out this formula explicitly each time.
A partial list of resulting issues follows:
\begin{itemize}
  \item Generally, the programmer must write more code if abstraction is not possible, meaning more time must be spent coding.
  \item Repeating the same code over and over can lead to careless translation errors, simply because it is so tedious.
    For example, if someone mistypes 9 somewhere as 8, the program will not work as intended.
    This sort of error can be very hard to catch.
    My own master's thesis (\url{https://ritdml.rit.edu/handle/1850/14123}) showed that in contexts where abstraction is not possible, people will resort to copying and pasting, and errors will happen as a result.
  \item Even if code is repeated correctly, it makes the code fragile to change.
    For example, say a programmer retroactively wants to add something to a formula.
    If abstraction had been in place, this change would be needed only once.
    Without abstraction, the change must be done everywhere the formula was used.
    Not only is this tedious, simply figuring out where the formula is used can be a challenge for large programs
  \item Without abstraction, the programmer generally needs to think about more information at once.
    For example, with the {\tt toCelcius} abstraction in place, the programmer could treat {\tt toCelcius} as a \emph{black box}: exactly how {\tt toCelcius} worked was unimportant, only that it did work.
    Without this abstraction, the programmer is forced to read the entire program at once.
    Abstraction is a basic necessity to allow the act of programming to scale.
\end{itemize}

To address these problems, we will add abstraction to our calculator language.
This is actually a very vague statement, as the term ``abstraction'' refers to a wide variety of things.
In fact, many of the differences between programming languages and even programming styles can be chalked up to how abstractions differ between them.
Rather than try to touch on all these points at once, we will start of small, and build our way up to more advanced abstraction mechanisms.

\section{Macros}
As a first step, we will explore the addition of \emph{macros} to our language.
A macro is an automatic textual substitution mechanism which will allow us to write the abstracted {\tt toCelcius} formula discussed in Section~\ref{sec:intro}.
The textual substitution will occur right before the program is evaluated, in a phase commonly known as \emph{macro expansion time}.
After macro expansion is complete, the resulting program still works with our old calculator syntax and semantics, so we do not have to change these at all.

To better understand what this means, consider the macro definition and program below:

\begin{flalign*}
&{\tt foo}(x, y) = x + y&
\\
&{\tt foo}(7, 8) + {\tt foo}(9, 10)&
\end{flalign*}

The program above defines a macro named {\tt foo} which takes two arguments $x$ and $y$.
It then uses {\tt foo} on the arguments 7 and 8, along with the arguments 9 and 10 on a second application of {\tt foo}.

After macro expansion finishes, the following program results:

\begin{gather*}
7 + 8 + 9 + 10
\end{gather*}

\ldots which clearly works with our existing calculator language definitions.
\end{document}
